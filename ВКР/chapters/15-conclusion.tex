\section*{Заключение}
\addcontentsline{toc}{section}{Заключение}

По сей день задача отслеживания объектов на фото- и видеоматериалах, полученных с беспилотных летательных аппаратов, остается одной из наиболее трудоемких областей применения нейросетевых алгоритмов. Несмотря на множество недавних исследований, подвигнутых появлением обширных датасетов и соревнований соответствующей тематики, существующие модели все еще имеют значительные ограничения и не позволяют решать поставленную задачу в полной мере.

В случае дальнейших исследований в направлении отслеживания объектов вызывает интерес вопрос о межкамерном трекинге \cite{16-1}, используемом для построения траектории объекта по нескольким камерам видеонаблюдения, которые имеют непересекающиеся области обзора. Адаптирование данной задачи под реалии группы беспилотных летательных аппаратов \cite{16-2} в перспективе может существенно повысить качество результатов в данной области.
