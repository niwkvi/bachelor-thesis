\subsection{Обзор существующих решений}

В статье \cite{14-1}, опубликованной по результатам соревнования VisDrone-MOT2021, представлены наилучшие решения, большинство из которых основаны на базе модели детектирования. В ведущих решениях используется множество передовых технологий, таких как Cascade R-CNN и CenterNet для детектирования объектов, а также DeepSORT, IOU и FairMOT для трекинга.

\textbf{Simple Online Multi-Object Tracker (SOMOT).} Модель детектирования основана на Cascade R-CNN, предобученной на датасете COCO, и embedding модель -- на Multiple Granularity Network (MGN) \cite{14-2}. Для этапа ассоциации построен трекинг множества объектов, использующий идеи DeepSORT и FairMOT, в начале инициализирующий множество треклетов, основываясь на полученных ограничивающих рамках первого кадра, а далее соотносящий новые рамки с треклетами в соответствии с вычисленным расстоянием между ними с учетом особенностей внешнего вида каждого треклета.

\textbf{GIAOTracker-Fusion.} Модель, состоящая из базового трекера, глобального трекера и пост-трекера. Базовая часть использует DetectoRS \cite{14-3} в качестве модели детектирования, предварительно обученную на датасете COCO и дообученную на VisDrone2019MOT, а в качестве трекера используется алгоритм DeepSORT. В целях стабилизации изображений из-за влияния движения камеры используется ORB+RANSAC, объединяющий механизм feature bank из DeepSORT, алгоритм обновления признаков из FairMOT и оптимизацию алгоритма Калмана. Также используется более мощное извлечение признаков при помощи OSNet \cite{14-4}. Глобальный трекер основан на треклетах из базового трекера, для извлечения признаков и ассоциаций которых используется модель VideoReID. Пост-трекер основан на результатах работы алгоритма глобального трекера, а также использует постобработку, такую как интерполяция и устранение шума.

\textbf{Multi-Object Tracking Approach Based on DeepSORT (MMDS).} В качестве базового детектора используется DetectoRS, а для трекинга применяется улучшенный DeepSORT. Для уменьшения влияния движения камеры используется Enhanced Correlation Coefficient Maximization, в то время как для сопоставления объектов происходит вычисление матрицы гомографии между соседними кадрами. Вместо стандартного фильтра Калмана используется Unscented Kalman Filter (UKF) для более точной оценки положений объектов во время движения. К тому же устанавливается правило, по которому отслеживаемые объекты, не совпадающие в течение $k$ кадров, не могут быть ассоциированы с объектами на текущем кадре, где значение $k$ меняется в зависимости от параметров треклетов. Последним из улучшений является то, что до применения немаксимального подавления сначала выполняется трекинг всех объектов и удаление пересекающихся траекторий.

\textbf{Deep IOU Tracker.} В качестве модели детектирования используется ансамбль нескольких моделей Cascade R-CNN, обученных при разных конфигурациях. При детектировании изображения разделяются на несколько частей для улучшения обнаружения мелких объектов. Ассоциация происходит через венгерский алгоритм, используется IoU между ограничивающими рамками вместе с косинусным сходством ReID признаков. Для решения проблемы, связанной с движением камеры, применяется Enhanced Correlation Coefficient Maximization.

\textbf{Implementation of DeepSORT with Scaled-YOLOv4 for Visual Drone Multi-Object Tracking (YOLO-DeepSORT-VisDrone).} В основе модели детектирования лежит улучшенная версия YOLOv4 -- Scaled-YOLOv4, предобученная на датасете COCO и обученная на VisDrone2021DET. Для трекинга используется DeepSORT.

\textbf{High-Speed Online Multi-Class Multi-Object Tracking with Center Point Based Cascaded Filtering (CenterPointCF).} Для достижения высокой скорости работы при одновременном трекинге объектов всех классов используется информация о центральных точках расположений объектов и оценке точности детектора CenterNET \cite{14-5}. Для построения легковесной модели состояний объектов используется фильтр Gaussian Mixture Probability Hypothesis Density (GMPHD) \cite{14-6}, являющийся решением для рекурсивной байесовской фильтрации.

\textbf{Medianflow-IoU-YOLO-Tracker (MIYoT).} Из соображений того, что эффективность модели детектирования значительно влияет на результативность трекинга, используется одна из последних моделей детектирования YOLO наряду с IoU и визуальным трекером. Изначально выполняется процесс ассоциации на основе IoU и активируется визуальный трекер для отслеживания объектов с предыдущих кадров, которые не были сопоставлены ни с одним объектом до этого. В течение этого процесса трекинг возвращается от визуального трекера к IoU алгоритму, если обнаруживается подходящее совпадение с одним из тех объектов, которые не были найдены ранее.

\textbf{HNet.} \cite{14-7} Подход основан на идее совместного детектирования и отслеживания объектов. Метод использует обнаружение центра и предсказание смещения центра при движении. Вдохновлен CenterTrack, однако в отличие от него применяется другой процесс построения тепловой карты, использующий метод селективной реконструкции центра. Модель принимает на вход предыдущий и текущий кадры, а также заранее преобразованную версию тепловой карты предыдущего кадра. На выходе получается тепловая карта, содержащая два гауссовских пика, представляющих местоположения центров. Наряду с этим модель предсказывает ограничивающие рамки и вектор движения для каждого объекта.
