\section*{Введение}
\addcontentsline{toc}{section}{Введение}

В современном мире беспилотные летательные аппараты стали неотъемлемой частью многих сфер человеческой деятельности благодаря своей универсальности и эффективности в применении \cite{1-1}, что позволило им стать мощным инструментом в различных прикладных задачах, таких как картография \cite{1-2}, система предупреждения лесных пожаров \cite{1-3}, система оперативного спасения людей \cite{1-4} и содействие в контроле чрезвычайных ситуаций \cite{1-5}.

Однако использованию беспилотных летательных аппаратов в вышеперечисленных областях сопутствует множество проблем, нестандартных для работы с видеоматериалами, полученными с помощью статичных камер видеонаблюдения, таких как вызванное движением камер размытие, помехи в сигналах передачи, переменные погодные условия и варьирующиеся размеры объектов, связанные с изменением высоты точки обзора \cite{1-6}.

В целях полного раскрытия потенциала использования беспилотных летательных аппаратов и расширения сфер их применения разрабатывается множество алгоритмов и моделей в области искусственного интеллекта \cite{1-7}, активно использующегося в задачах поиска, классификации и трекинга объектов как на записях с камер, так и в условиях реального времени.

Первоначальным этапом трекинга является детектирование и классификация, заключающиеся в выделении на каждом отдельном кадре объектов, например, ограничивающими рамками, и определении классов, к которым они принадлежат. Некоторыми из известных и пользующихся популярностью решений данной задачи являются R-CNN \cite{1-8} и версия для более быстрых вычислений FAST R-CNN \cite{1-9}, их модификация FASTER R-CNN \cite{1-10}, использующая для генерации областей возможного наличия объектов специальную нейронную сеть RPN (Region Proposal Network), Mask R-CNN \cite{1-11} и TensorMask \cite{1-12}, позволяющие решать задачу сегментации, а также RetinaNet \cite{1-13} и YOLO \cite{1-14}, отличительной особенностью которых является их одностадийная архитектура.

Следующим этапом данной задачи служат алгоритмы трекинга, позволяющие сопоставлять объекты и определять взаимосвязь их расположений на соседних кадрах из общей последовательности, что позволяет устанавливать для каждого объекта на видео его уникальный идентификационный номер, что, в свою очередь, делает возможным подсчет объектов на видео. Возникающая проблема сопоставления траекторий таких объектов может решаться применением алгоритмов трекинга, таких как SORT \cite{1-15} и его улучшения при помощи реидентифицации объектов по их внешнему виду DeepSORT \cite{1-16}, TransMOT \cite{1-17}, представляющим упорядоченные траектории всех отслеживаемых объектов в виде серии разреженных взвешенных графов, построенных с использованием пространственных отношений между целями, а также StrongSORT \cite{1-18} и ByteTrack \cite{1-19}.

Дополнительным этапом может служить постобработка, позволяющая улучшить работу алгоритма трекинга и повысить точность результатов, например, при помощи сглаживания траекторий и дополнения пропущенных моментов движения. Примерами такого механизма улучшения трекинга являются фильтр Калмана \cite{1-20}, фильтр частиц \cite{1-21} и сегментация \cite{1-22}.
