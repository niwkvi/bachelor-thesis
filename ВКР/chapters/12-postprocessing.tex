\subsection{Постобработка и полученные результаты}

Важно заметить, что в результате работы алгоритма трекинга StrongSORT прослеживается нестабильный характер расположения ограничивающих рамок вдоль траекторий объектов по причине движения камер беспилотных летательных аппаратов.

Для устранения данной проблемы и улучшения полученных после трекинга результатов к траекториям была применена линейная интерполяция, позволяющая восстанавливать пропадающие ограничивающие рамки объектов, а также гауссовское сглаживание, обеспечивающее плавность траекторий.

После применения вышеописанных модификаций к модели при помощи инструментов, предоставленных организаторами соревнования для датасета VisDrone, был получен $AP \ 56.45$, представленный в сравнительной Таблице \ref{tbl:12-1}.

\vspace{0.5cm}

\begin{table}[ht]
    \centering
    \begin{tblr}{c|c|c|c|c}
        \hline[1.5pt]
        Алгоритм & $AP$ & $AP_{0.25}$ & $AP_{0.5}$ & $AP_{0.75}$ \\
        \hline[1.5pt]
        SOMOT & \textbf{{\color{myred}58.61}} & \textbf{{\color{myred}70.75}} & \textbf{{\color{myred}61.26}} & \textbf{{\color{myred}43.84}} \\
        \hline
        YOLOv5-StrongSORT-GSI & \textbf{\color{mygreen}56.45} & \textbf{\color{mygreen}67.83} & \textbf{\color{mygreen}57.13} & \textbf{\color{myblue}42.35} \\
        \hline
        GIAOTracker-Fusion & \textbf{\color{myblue}54.18} & \textbf{\color{myblue}63.41} & \textbf{\color{myblue}55.35} & \textbf{\color{mygreen}43.78} \\
        \hline
        MMDS & 52.68 & 62.92 & 53.42 & 41.69 \\
        \hline
        Deep IoU Tracker & 48.54 & 63.16 & 48.11 & 34.33 \\
        \hline
        YOLO-DeepSORT-VisDrone & 46.70 & 57.43 & 48.92 & 33.75 \\
        \hline
        CenterPointCF & 44.03 & 56.91 & 44.09 & 31.09 \\
        \hline
        MIYoT & 39.35 & 50.72 & 39.25 & 28.10 \\
        \hline
        HNet & 24.71 & 33.88 & 24.35 & 15.89 \\
        \hline[1.5pt]
    \end{tblr}
    \caption{Сравнение полученных результатов с известными решениями}
    \label{tbl:12-1}
\end{table}
